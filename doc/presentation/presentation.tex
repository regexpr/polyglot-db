\documentclass[12pt,usenames,dvipsnames]{beamer}

%\usetheme[progressbar=frametitle]{metropolis}
\usetheme[subsectionpage=progressbar]{metropolis}

\metroset{block=fill}
%\begin{itemize}[<+- | alert@+>]


\usepackage[utf8]{inputenc}
% Für Häkchen
\usepackage{ pifont }
\newcommand{\cmark}{\ding{51}}%
\newcommand{\xmark}{\ding{55}}%
% Farben
\usepackage{xcolor}

\usepackage{graphicx}
\graphicspath{ {images/} }

\setbeamertemplate{itemize item}{\tiny$\bullet$}
\setbeamertemplate{itemize subitem}{\tiny$\bullet$}

\title{PolyGlot-Database Performance}
\subtitle{Framework zum Vergleichen von Anfragedauern auf MongoDB und Neo4J}
\author{Hyeon Ung Kim, Tim Niehoff}
%\institute{}
\date{6. August 2018}

\makeatletter
\setbeamertemplate{section page}{
  \centering
  \begin{minipage}{22em}
    \raggedright
    \usebeamercolor[fg]{section title}
    \usebeamerfont{section title}
    \centering\insertsectionhead\\[-1ex]
    \centering\usebeamertemplate*{progress bar in section page}
    \par
    \ifx\insertsubsectionhead\@empty\else%
      \usebeamercolor[fg]{subsection title}%
      \usebeamerfont{subsection title}%
      \centering\insertsubsectionhead
    \fi
  \end{minipage}
  \par
  \vspace{\baselineskip}
}
\makeatother

\begin{document}

	\maketitle
			
%	\begin{frame}{Gliederung}
%		\setbeamertemplate{section in toc}[sections numbered]
%\setbeamertemplate{subsection in toc}[subsections numbered]
%		\tableofcontents
%	\end{frame}
	\section{Aufgabenstellung}
	\begin{frame}{Heranführung}
	\begin{itemize}[<+- | alert@+>]
	\item Verschiedene Anwendungen erfordern versch. Typen
von Datenbanken: Relational, Key-Value, Document, Graph\footnote{ Inhalt der Folie von \url{https://dbs.uni-leipzig.de/file/Intro_bdprak_final.pdf}}
	\item In der Praxis: Gleichzeitige Verwendung versch. Typen
	\item Vorteil: Optimale DB für jeden Anwendungsfall
	\begin{itemize} [<+- | alert@+>]
	\item Relational: Sicherheit, homogene Daten
	\item Document: Flexibles Schema, Suchfunktionen
	\item Graph: Beziehungen, Traversal
	\end{itemize}
	\item Aufgabe: Welchen Vorteil hat die Verwendung einer
Graphdatenbank gegenüber einer Dokumenten-Datenbank?
	\end{itemize}	
	\end{frame}
		\begin{frame}{gegebene Werkzeuge}
		\begin{table}[]
\begin{tabular}{ll}
\includegraphics[width=0.2\textwidth]{java}  & \includegraphics[width=0.3\textwidth]{yelp} \\
\includegraphics[width=0.4\textwidth]{mongodb}  & \includegraphics[width=0.35\textwidth]{neo4j}
\end{tabular}
\end{table}
	\end{frame}
\begin{frame}{Stichpunkte zur vorigen Folie}
\begin{itemize}[<+- | alert@+>]
\item Mongo + Neo4j Driver
\item teilweise nicht ausgereift. Für das garantieren von beliebigen Queries muss geparst werden

\end{itemize}

\end{frame}
\section{Problem: Gleicher Datensatz auf beiden DB?}
\subsection{Mongo Connector + Neo4j Doc Manager}
\begin{frame}{Mongo Connector + Neo4j Doc Manager}
\begin{itemize}
\item Softwarelösung von MongoDB und Neo4j
\item synchronisiert mithilfe von Mongo Replica Sets von Mongo in Neo4j
\item muss kein Datenmodell angegeben werden. Funzt Generisch
\end{itemize}
\end{frame}
\subsection{Apoc}
\begin{frame}

\begin{itemize}[<+- | alert@+>]
\item Software von Neo4j zum Import von JSONs in die Neo4j
\item keine Synchronisierung zwischen Mongo und Neo4j
\item Nutzer muss Neo4j Datenmodell erstellen
\item durch nutzerspezifisches Datenmodell kann man die Graphdatenbankvorteile auskosten
\end{itemize}

\end{frame}
\section{Resultat: PolyGDBP}
\subsection{Pipline und Workflow}
\begin{frame}
\begin{itemize}[<+- | alert@+>]
\item File Struktur (Main, Mongo, Neo4j, Benchmark)
\item Command line Interface
\item nur Queryangabe ist pflicht, alles andere optional.
\item gibt vorgefertigte Queries für Yelp Datensatz
\item Nutzer kann JSON Datensatz importieren
\item Wird in MongoCollections verarbeitet und durch MongoConnector erfolgt danach das Übertragen in Neo4j
\item Queries werden ausgeführt
\item vor und jedem dieser Schritte zeit gestoppt, geloggt und schließlich ausgegeben
\item kann mithilfe unserer Javascript Anwendung visualisiert werden.
\end{itemize}
\end{frame}
\section{Lets run it. With the help of YELP}
\subsection{Datensatz + DatenModell}
\begin{frame}
\begin{itemize}[<+- | alert@+>]
\item YELP = Suchmaschine und Empfehlungsportal für Restaurants und Geschäfte 
\item YELP Datensatz X GB groß
\item besteht aus X .jsons: users, business, review, ...
\item Beispiele eines JSON Business Objektes:
\item ...
\end{itemize}
\end{frame}
\begin{frame}{Neo4j Datenmodell}
\includegraphics[width=1\textwidth]{neo4jModell}

\end{frame}
\subsection{Ergebnisse}
	\section{Zusammenfassung}
	\begin{frame}{Zusammenfassung}
	\begin{itemize}[<+- | alert@+>]
	\item Verschiedene DB-Typen haben versch. Vor- und Nachteile
	\item PolyG-DBP ist ein Framework zum Vergleich einer Dokument-DB mit einer Graphdatenbank
	\item konkret: MongoDB und Neo4j werden getestet
	\item Nutzer kann prebuilt queries ausführen lassen oder eigene Queries testen lassen
	\item Ausführungszeiten der Queries werden gemessen und verglichen
	\item Unsere Tests mithilfe von PolyG-DBP und Yelp zeigen: Neo4j und MongoDB siegen bei bestimmten Arten von Queries
	\end{itemize}
	\end{frame}
\begin{frame}[standout]
  Fragen und Diskussion
\end{frame}


\end{document}